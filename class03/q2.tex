\section{Aula 03 - Questão 2}

2. Crie instâncias aleatórias com idades (inteiros) variando entre 0 e 100, com tamanhos \( n = 100 \), \( n = 1000 \), \( n = 1000000 \).

\begin{verbatim}
#include <iostream>
#include <vector>
#include <algorithm>
#include <random>
#include <chrono>

bool idadeRep(const std::vector<int>& Idade) {
    int tam = Idade.size();
    if (tam == 0) return false;

    int menor = 200;  
    for (int i = 0; i < tam; i++) {
        if (Idade[i] < menor) {
            menor = Idade[i];
        }
    }

    int count = 0;
    for (int i = 0; i < tam; i++) {
        if (Idade[i] == menor) {
            count++;
        }
    }

    return count > 1; 
}

bool idadeRep2(std::vector<int> Idade) {
    std::sort(Idade.begin(), Idade.end());
    return Idade[0] == Idade[1];
}

std::vector<int> generateRandomAges(int n) {
    std::vector<int> ages(n);
    std::default_random_engine generator;
    std::uniform_int_distribution<int> distribution(0, 100);

    for (int i = 0; i < n; i++) {
        ages[i] = distribution(generator);
    }
    return ages;
}

int main() {
    int sizes[] = {100, 1000, 1000000};

    for (int n : sizes) {
        std::vector<int> randomAges = generateRandomAges(n);
        
        auto start = std::chrono::high_resolution_clock::now();
        idadeRep(randomAges);
        auto end = std::chrono::high_resolution_clock::now();
        std::chrono::duration<double, std::micro> duration1 = end - start;

        start = std::chrono::high_resolution_clock::now();
        idadeRep2(randomAges);
        end = std::chrono::high_resolution_clock::now();
        std::chrono::duration<double, std::micro> duration2 = end - start;

        std::cout << "n = " << n << ": "
                  << "idadeRep: " << duration1.count() << " microsegundos, "
                  << "idadeRep2: " << duration2.count() << " microsegundos" 
                  << std::endl;
    }

    return 0;
}

\end{verbatim}