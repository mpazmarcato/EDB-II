\section{Aula 03 - Questão 1}

1. Implemente os códigos idadeRep e idadeRep2;

Códigos implementados em C++.


\begin{verbatim}
    bool idadeRep(const std::vector<int>& Idade) {
        int tam = Idade.size();
        if (tam == 0) return false; 
    
        int menor = 200;  
        for (int i = 0; i < tam; i++) {
            if (Idade[i] < menor) {
                menor = Idade[i];
            }
        }
    
        int count = 0;
        for (int i = 0; i < tam; i++) {
            if (Idade[i] == menor) {
                count++;  
            }
        }
    
        return count > 1; 
    }
    
    
    bool idadeRep2(vector<int> Idade){
        sort(Idade.begin(), Idade.end());
        return Idade[0] == Idade[1];
    }

\end{verbatim}