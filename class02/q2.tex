\section{Aula 02 - Questão 2}

2. Qual a complexidade média no caso do algoritmo2?

A complexidade média da busca binária, assim como a complexidade no pior caso, é O(log n), onde n é o número de elementos na lista.

Isso ocorre porque a busca binária sempre divide a lista ao meio, independentemente de onde o elemento procurado está (se está no início, meio ou final da lista). A busca binária funciona com uma estratégia de divisão e conquista, reduzindo o tamanho do problema pela metade a cada passo. Assim, o número de passos necessários para encontrar o elemento é proporcional ao logaritmo da quantidade de elementos.

Portanto, em um cenário médio, onde não temos informações sobre a posição exata do elemento, a complexidade continua sendo O(log n), pois, em média, o algoritmo precisará realizar o mesmo número de divisões até encontrar o elemento ou concluir que ele não está presente.