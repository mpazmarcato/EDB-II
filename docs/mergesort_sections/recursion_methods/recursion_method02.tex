\subsubsection{Método da Iteração}
Seja a função recursiva da mergesort definida por \( T(n) = 2T\left(\frac{n}{2}\right) + n \). Utilizando o método da iteração, iremos encontrar seu nível de complexidade. 

Expandindo a recorrência:

\begin{align*}
i = 1 : &\quad T(n) = 2T\left(\frac{n}{2}\right) + n \\
i = 2 : &\quad T(n) = 2\left(2T\left(\frac{n}{4}\right) + \frac{n}{2}\right) + n = 4T\left(\frac{n}{4}\right) + n + n = 4T\left(\frac{n}{4}\right) + 2n\\
i = 3 : &\quad T(n) = 4\left(2T\left(\frac{n}{8}\right) + \frac{n}{4}\right) + 2n = 8T\left(\frac{n}{8}\right) + n + 2n = 8T\left(\frac{n}{8}\right) + 3n\\
\end{align*}

De forma geral, temos:

\[
T(n) = 2^i T\left(\frac{n}{2^i}\right) + i \cdot n
\]

A fim de alcançar o caso base, calculamos:

\[
\frac{n}{2^i} = 1 \implies 2^i = n \implies i = \log_{2}{n}
\]

Substituindo, obtemos:

\[
T(n) = 2^{\log_{2}{n}} T(1) + n \cdot \log_{2}{n}
\]

Manipulando, temos:

\[
T(n) = 2^{\log_{2}{n}} T(1) + n \cdot \log_{2}{n} = n \cdot 1 + n \cdot \log_{2}{n}
\]

Note que a complexidade que se sobrepõe é a de \( n \cdot \log_{2}{n} \). Logo, a relação de recorrência tem complexidade de \( O(n \log n) \).
