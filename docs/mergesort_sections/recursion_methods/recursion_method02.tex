\subsubsection{Método da Iteração}
Seja a função recursiva da mergesort definida por $T(n) = 2T(\frac{n}{2}) + n$, utilizando o método da iteração, irei encontrar seu nível de complexidade. \\
Expandindo a recorrência:\\
$i = 1 : 2T(\frac{n}{2}) + n$ \\
$i = 2 : 2(2T(\frac{n}{4}) + \frac{n}{2}) + n = 4T(\frac{n}{4}) + n + n = 4T(\frac{n}{4}) + 2n$ \\
$i = 3 : 4(2T(\frac{n}{8}) + \frac{n}{4}) + 2n = 8T(\frac{n}{8}) + n + 2n = 8T(\frac{n}{8}) + 3n$ \\
$i : 2^iT(\frac{n}{2^i}) + i\cdot n$ \\
A fim de alcançar o caso base, calculemos: \\
$\frac{n}{2^i} = 1 \implies 2^i = n \implies k = \log_{2}{n}$ \\
Substituindo, $2^iT(\frac{n}{2^i}) + i\cdot n = 2^{\log_{2}{n}}T(1) + n \cdot \log_{2}{n}$ \\
Manipulando: \\
$2^{\log_{2}{n}}T(1) + n \cdot \log_{2}{n} = n \cdot 1 + n \cdot \log_{2}{n}$ \\
Note que, a complexidade que se sobrepõe é a de $n\cdot \log_{2}{n}$, Logo, a relação de recorrência tem complexidade de $O(n\log{n})$.