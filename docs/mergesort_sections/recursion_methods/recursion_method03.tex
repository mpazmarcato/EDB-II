\subsubsection{Método da Árvore de Recursão}
A relação de recorrência da mergesort recursiva é dada por $T(n) = 2T(n/2) + n$. Utilizando o método de árvore de recorrência, temos:

\begin{table}[ht!]
    \centering
    \begin{tabular}{|c|c|c|c|}
    \hline
    \textbf{Nível} & \textbf{Tamanho} & $\#$\textbf{Nós} & \textbf{Tempo} \\ \hline
     0 & $n$ & 1 & $n$ \\ \hline
     1 & $n/2$ & 2 & $n/2$ \\ \hline
     2 & $n/4$ & 4 & $n/4$ \\ \hline
     3 & $n/8$ & 8 & $n/8$ \\ \hline
     (...) & & & \\ \hline
     $i$ & $n/2^i$ & $2^i$ & $n/2^i$ \\ \hline 
    \end{tabular}  
    \caption{Árvore de recursão}
\end{table}

Agora, calculando o somatório do tempo multiplicado pelo número de nós:

\[
\frac{n}{2^i} = 1 \implies n = 2^i \implies i = \log_2{n}
\]

Logo, o somatório é:

\[
\sum_{i=0}^{\log_2{n}} \left( 2^i \times \frac{n}{2^i} \right) = \sum_{i=0}^{\log_2{n}} n = n \log_2{n}
\]

Portanto, a complexidade é $\Theta(n \log n)$.
