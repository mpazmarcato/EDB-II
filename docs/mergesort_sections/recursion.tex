\subsection{Recursiva}
\begin{verbatim}
# Realiza a ordenação de um vetor desordenado com a estratégia do MergeSort.
mergeSort(arr, inicio, fim) {
    Se (inicio >= fim) Retornar

    meio = (inicio + fim) / 2
    mergeSort(arr, inicio, meio)
    mergeSort(arr, meio + 1, fim)
    merge(arr, inicio, meio, fim)
}

# Função para mesclar duas sub-listas
funcao merge(vetor, inicio, meio, fim) {
    n1 = meio - inicio + 1 # Tamanho da primeira sub-lista
    n2 = fim - meio # Tamanho da segunda sub-lista

    # Vetores auxiliares para armazenar as sub-listas
    L: vetor
    R: vetor

    # Copia os elementos para o vetor auxiliar L
    para (int i = 0; i < n1; i++) {
        L[i] = vetor[inicio + i]
    }

    # Copia os elementos para o vetor auxiliar R
    para (int i = 0; i < n2; i++) {
        R[j] = vetor[meio + 1 + j]
    }

    i = 0 # Índice para L
    j = 0 # Índice para R
    k = inicio # Índice para vetor

    # Mescla as duas sub-listas em vetor
    enquanto (i < n1 && j < n2) {
        se (L[i] <= R[j]) {
            vetor[k] = L[i] # Adiciona o menor elemento em vetor
            i = i + 1 # Move para o próximo elemento em L
        } senão {
            vetor[k] = R[j] # Adiciona o menor elemento em vetor
            j = j + 1 # Move para o próximo elemento em R
        }
        k = k + 1 # Move para o próximo índice em vetor
    }

    # Copia os elementos restantes de L, se houver
    enquanto (i < n1) {
        vetor[k] = L[i]
        i = i + 1
        k = k + 1
    }

    # Copia os elementos restantes de R, se houver
    enquanto (j < n2) {
        vetor[k] = R[j]
        j = j + 1
        k = k + 1
    }
}
\end{verbatim}