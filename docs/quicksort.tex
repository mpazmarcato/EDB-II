\section{QuickSort}
Abaixo estão os pseudocódigos do algoritmo \textbf{QuickSort}, em versões iterativa e recursiva. Há o uso da função auxiliar $particiona$, definida a seguir:
\begin{verbatim}
# Função para particionar um vetor em dois baseado em um pivô.
particiona(vetor, valorBaixo, valorAlto) { 
  # Obter o valor central relativo ao vetor.
  meio = valorBaixo + (valorAlto - valorBaixo) / 2
  pivo = vetor[meio] # Obtém o pivô.

  # Variáveis internas que controlam os índices em uso.
  inicio = valorBaixo
  final = valorAlto 
  
  # Percorre de uma extremidade à outra.
  enquanto(inicio <= final) { 
    # Ajusta o índice da extremidade esquerda até encontrar um elemento 
    # à esquerda maior que o pivô.
    enquanto (vetor[inicio] < pivo) { 
      inicio++
    }
    # Ajusta o índice da extremidade direita até encontrar um elemento 
    # à direita menor que o pivô.
    enquanto (vetor[final] > pivo) {
      final--
    }

    # Se os índices das extremidades não se encontraram, 
    # os valores maiores e menores que o pivô não estão
    # separados, fazendo necessária uma troca arbitrária.
    se (inicio <= final) {
      trocar(vetor, inicio, final)
      inicio++
      final--
    }
  }
  
  # Retorna a posição do novo pivô
  retorne inicio
}
   
# Função para trocar dois elementos em um vetor.
trocar(vetor, indice1, indice2) { 
  elementoAuxiliar = vetor[indice2]
  vetor[indice2] = vetor[indice1]
  vetor[indice1] = elementoAuxiliar
}
\end{verbatim}
\subsection{Iterativa}
\begin{verbatim}
# Realiza a ordenação de um vetor desordenado com a estratégia do MergeSort.
mergeSort(vetor) {
    n = tamanho(vetor) # Obtém o tamanho do vetor
    largura = 1 # Inicializa a largura das sub-listas

    # Enquanto a largura das sub-listas for menor que o tamanho do vetor
    enquanto (largura < n) {
        # Itera sobre o vetor em passos de 2 * largura
        para (int i = 0; i < n - 1; i += 2 * largura) {
            inicio = i # Início da sub-lista
            meio = min(i + largura - 1, n - 1) # Meio da sub-lista
            fim = min(i + 2 * largura - 1, n - 1) # Fim da sub-lista

            # Chama a função de mesclagem para combinar as sub-listas
            merge(vetor, inicio, meio, fim)
        }

        # Dobra a largura para a próxima iteração
        largura = largura * 2
    }
}

# Função para mesclar duas sub-listas
funcao merge(vetor, inicio, meio, fim) {
    n1 = meio - inicio + 1 # Tamanho da primeira sub-lista
    n2 = fim - meio # Tamanho da segunda sub-lista

    # Vetores auxiliares para armazenar as sub-listas
    L: vetor
    R: vetor

    # Copia os elementos para o vetor auxiliar L
    para (int i = 0; i < n1; i++) {
        L[i] = vetor[inicio + i]
    }

    # Copia os elementos para o vetor auxiliar R
    para (int i = 0; i < n2; i++) {
        R[j] = vetor[meio + 1 + j]
    }

    i = 0 # Índice para L
    j = 0 # Índice para R
    k = inicio # Índice para vetor

    # Mescla as duas sub-listas em vetor
    enquanto (i < n1 && j < n2) {
        se (L[i] <= R[j]) {
            vetor[k] = L[i] # Adiciona o menor elemento em vetor
            i = i + 1 # Move para o próximo elemento em L
        } senão {
            vetor[k] = R[j] # Adiciona o menor elemento em vetor
            j = j + 1 # Move para o próximo elemento em R
        }
        k = k + 1 # Move para o próximo índice em vetor
    }

    # Copia os elementos restantes de L, se houver
    enquanto (i < n1) {
        vetor[k] = L[i]
        i = i + 1
        k = k + 1
    }

    # Copia os elementos restantes de R, se houver
    enquanto (j < n2) {
        vetor[k] = R[j]
        j = j + 1
        k = k + 1
    }
}

\end{verbatim}
O algoritmo iterativo tem diferentes funções de complexidade para cada caso:
\begin{itemize}
  \item Melhor caso: ocorre quando a lista está ordenada; o loop interno será executado $n$ vezes e o externo, somente 1. Portanto, temos uma complexidade expressa por algo como $f(n) = n + 3$, onde $n$ representa o loop \textit{for} e o 3, as instruções que não dependem da entrada.
  \item Pior caso: ocorre quando a lista está completamente desordenada; o loop externo ocorre $n$ vezes e também o interno. Desse modo, temos uma complexidade expressa por algo como $g(n) = n^2 + 5$, onde 5 representa o total de instruções que não dependem da entrada e $n^2$ a quantidade de iterações total feito pelos loops.
\end{itemize}

Note que, para o melhor caso, o algoritmo tem complexidade $\Theta(n)$, pois:
\begin{align*}
    \lim_{n\to\infty} \frac{f(n)}{n} &= \lim_{n\to\infty} \frac{n + 3}{n} \\
    &= \lim_{n\to\infty} 1 + \frac{3}{n} \\ 
    &= 1 
\end{align*}

Adicionalmente, para o pior caso, o algoritmo tem complexidade $\Theta(n^2)$, pois:
\begin{align*}
    \lim_{n\to\infty} \frac{g(n)}{n} &= \lim_{n\to\infty} \frac{n^2 + 5}{n^2} \\
    &= \lim_{n\to\infty} 1 + \frac{5}{n^2} \\ 
    &= 1 
\end{align*}

Portanto, o algoritmo tem $\Omega(n)$, $O(n^2)$ e não há $\Theta$.


\subsection{Recursiva}
\begin{verbatim}

# Realiza a ordenação de um vetor desordenado com o algoritmo Quick Sort.
quicksort(vetor, valorEsquerda, valorDireita) { 
  if (valorEsquerda < valorDireita) {
    # Pivô é o ponto onde o vetor será dividido em duas partes.
    pivo = particiona(vetor, valorEsquerda, valorDireita)
    # Realiza a ordenação no subvetor à esquerda do pivô.
    quicksort(vetor, valorEsquerda, pivo)
    # Realiza a ordenação no subvetor à direita do pivô.
    quicksort(vetor, pivo + 1, valorDireita) 
  }
}

\end{verbatim}


O algoritmo recursivo tem como função de complexidade $T(n) = 2T(\frac{n}{2}) + n$. Resolveremos a recorrência pelos métodos abaixo:
\subsubsection{Método da iteração}
Seja a função recursiva do pior caso da bubblesort definida por $T(n) = T(n-1) + n$, utilizando o método da iteração, irei encontrar seu nível de complexidade. \\
Expandindo a recorrência: \\
$i = 1 : T(n - 1) + n$ \\
$i = 2 : (T(n - 2) + (n -1)) + n$ \\
$i = 3 : (T(n - 3) + (n - 2) + (n - 1)) + n$ \\
$i : T(n - i) + ... + (n - 1) + n$ \\
$1 + \sum_{i=2}^{n}k = 1 + (\frac{n(n-1)}{2} - 1)$ \\
$= \frac{n(n-1)}{2}$ \\
$= \frac{n^2 - n}{2}$ \\
Note que, a complexidade que se sobrepõe é a de $n^2$, Logo, a relação de recorrência tem complexidade de $O(n^2)$.
\subsubsection{Metodo da Árvore de Recursão}
Relação de Recorrência da quicksort recursiva para o melhor caso, $T(n) = 2T(n/2) + n$. Utilizando o método árvore de recorrência.

\begin{table}[ht!]
    \centering
    \begin{tabular}{|c|c|c|c|}
    \hline
    \textbf{Nível} & \textbf{Tamanho} & $\#$\textbf{Nós} & \textbf{Tempo} \\ \hline
     0 & n & 1 & n \\ \hline
     1 & $n/2$ & 2 & $n/2$ \\ \hline
     2 & $n/4$ & 4 & $n/4$ \\ \hline
     3 & $n/8$ & 8 & $n/8$ \\ \hline
     (...) & & & \\ \hline
     i & $n/2^i$ & $2^i$ & $n/2^i$ \\ \hline 
    \end{tabular}  
    \caption{Árvore de recursão}
\end{table}

Calculando o somatório do tempo x nós. \\
$n/2^i = 1 \rightarrow n = 2^i \rightarrow i = \log_2{n}$ \\
$\sum_{i=0}^{\log_2{n}} (2^i \times n/2^i) = \sum_{i=0}^{\log_2{n}} n = n \log_2{n}$. Logo, ele é $\Theta n \log{n}$ 

Relação de Recorrência da quicksort recursiva para o pior caso, $T(n) = T(n - 1) + n$. Utilizando o método árvore de recorrência.

\begin{table}[ht!]
    \centering
    \begin{tabular}{|c|c|c|c|}
    \hline
    \textbf{Nível} & \textbf{Tamanho} & $\#$\textbf{Nós} & \textbf{Tempo} \\ \hline
     0 & n & 1 & n \\ \hline
     1 & $n - 1$ & 1 & $n/2$ \\ \hline
     2 & $n - 2$ & 1 & $n/4$ \\ \hline
     3 & $n - 3$ & 1 & $n/8$ \\ \hline
     (...) & & & \\ \hline
     i & $n - i$ & $1$ & $n/2^i$ \\ \hline 
    \end{tabular}  
    \caption{Árvore de recursão}
\end{table}

Calculando o somatório do tempo x nós. \\
$1 = n -i \rightarrow i = n - 1$ \\
$\sum_{i = 0}^{n - 1} (n/2^i) = 2n(1 - 1/2^n)$. Logo, ele é $\Theta n$, pois o $2^n$ está tendendo a 0, tornando o n a complexidade predominante.

\subsubsection{Método do Teorema Mestre}
O método do teorema mestre não é aplicável ao algoritmo do Bubble Sort, já que sua função de complexidade $T(n)$ não está no padrão $aT(\frac{n}{b}) + \Theta(n^k)$.



