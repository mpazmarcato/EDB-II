\section{QuickSort}
Abaixo estão os pseudocódigos do algoritmo \textbf{QuickSort}, em versões iterativa e recursiva. Há o uso da função auxiliar $particiona$, definida a seguir:
\begin{verbatim}
# Função para particionar um vetor em dois baseado em um pivô.
particiona(vetor, valorBaixo, valorAlto) { 
  # Obter o valor central relativo ao vetor.
  meio = valorBaixo + (valorAlto - valorBaixo) / 2
  pivo = vetor[meio] # Obtém o pivô.

  # Variáveis internas que controlam os índices em uso.
  inicio = valorBaixo
  final = valorAlto 
  
  # Percorre de uma extremidade à outra.
  enquanto(inicio <= final) { 
    # Ajusta o índice da extremidade esquerda até encontrar um elemento 
    # à esquerda maior que o pivô.
    enquanto (vetor[inicio] < pivo) { 
      inicio++
    }
    # Ajusta o índice da extremidade direita até encontrar um elemento 
    # à direita menor que o pivô.
    enquanto (vetor[final] > pivo) {
      final--
    }

    # Se os índices das extremidades não se encontraram, 
    # os valores maiores e menores que o pivô não estão
    # separados, fazendo necessária uma troca arbitrária.
    se (inicio <= final) {
      trocar(vetor, inicio, final)
      inicio++
      final--
    }
  }
  
  # Retorna a posição do novo pivô
  retorne inicio
}
   
# Função para trocar dois elementos em um vetor.
trocar(vetor, indice1, indice2) { 
  elementoAuxiliar = vetor[indice2]
  vetor[indice2] = vetor[indice1]
  vetor[indice1] = elementoAuxiliar
}
\end{verbatim}
\subsection{Iterativa}
\begin{verbatim}

# Realiza a ordenação de um vetor desordenado com o algoritmo Quick Sort de forma iterativa.
quickSortIterative(vetor, valorEsquerda, valorDireita) {
    # Inicializa uma pilha para gerenciar os subvetores
    pilha = nova Pilha()
    
    # Empilha os limites iniciais do vetor
    pilha.empilhar((valorEsquerda, valorDireita))
    
    # Processa os subvetores até que a pilha esteja vazia
    enquanto (pilha não está vazia) {
        # Desempilha os limites atuais do subvetor
        (valorBaixo, valorAlto) = pilha.desempilhar()
        
        # Particiona o subvetor atual e obtém a posição do pivô
        pivo = particiona(vetor, valorBaixo, valorAlto)
        
        # Se houver elementos à esquerda do pivô, empilha o subvetor esquerdo
        se (valorBaixo < pivo - 1) {
            pilha.empilhar((valorBaixo, pivo - 1))
        }
        
        # Se houver elementos à direita do pivô, empilha o subvetor direito
        se (pivo < valorAlto) {
            pilha.empilhar((pivo, valorAlto))
        }
    }
}

# Função para particionar um vetor em dois baseado em um pivô.
particiona(vetor, valorBaixo, valorAlto) { 
  # Obter o valor central relativo ao vetor.
  meio = valorBaixo + (valorAlto - valorBaixo) / 2
  pivo = vetor[meio] # Obtém o pivô.

  # Variáveis internas que controlam os índices em uso.
  inicio = valorBaixo
  final = valorAlto 
  
  # Percorre de uma extremidade à outra.
  enquanto(inicio <= final) { 
    # Ajusta o índice da extremidade esquerda até encontrar um elemento 
    # à esquerda maior que o pivô.
    enquanto (vetor[inicio] < pivo) { 
      inicio++
    }
    # Ajusta o índice da extremidade direita até encontrar um elemento 
    # à direita menor que o pivô.
    enquanto (vetor[final] > pivo) {
      final--
    }

    # Se os índices das extremidades não se encontraram, 
    # os valores maiores e menores que o pivô não estão
    # separados, fazendo necessária uma troca arbitrária.
    se (inicio <= final) {
      trocar(vetor, inicio, final)
      inicio++
      final--
    }
  }
  
  # Retorna a posição do novo pivô
  retorne inicio
}
   
# Função para trocar dois elementos em um vetor.
trocar(vetor, indice1, indice2) { 
  elementoAuxiliar = vetor[indice2]
  vetor[indice2] = vetor[indice1]
  vetor[indice1] = elementoAuxiliar
}
    \end{verbatim}
\subsection{Recursiva}
\begin{verbatim}
# Realiza a ordenação de um vetor desordenado com a estratégia do Bubble Sort.
bubblesort_recursivo(lista, n){
    #lista: a lista a ser ordenada
    #n = o tamanho da lista

    # Caso base: se a lista tiver 1 ou 0 elementos, já está ordenada
    se n <= 1{
        retornar lista;
    }
    # Percorrer a lista comparando elementos adjacentes
    para (i de 0 até n-2){
        se(lista[i] > lista[i+1]){
            # Trocar elementos se estiverem na ordem errada
            trocar(lista[i], lista[i+1])
            }
    }
     # Chamar recursivamente a função para ordenar os n-1 primeiros elementos
    bubble_sort_recursivo(lista, n-1)
}    

# Faz a troca entre dois elementos de um vetor.
troca(vetor, indice1, indice2) { 
  elementoAuxiliar = vetor[indice2]
  vetor[indice2] = vetor[indice1]
  vetor[indice1] = elementoAuxiliar
}
\end{verbatim}

\subsection{Análise dos Algoritmos Recursivos}

<<<<<<<< HEAD:docs/quicksort_sections/recursion_methods/recursion_method02.tex
\subsubsection{Método da Iteração}
========
\subsubsection{Método 02 - Iterativo}
>>>>>>>> 9e4497fefbd53b11f69e25bf7f167bdcd9e3f9d4:docs/quicksort_sections/recursion_method02.tex
Relação de Recorrência da quicksort recursiva em seu melhor caso, $T(n) = 2T(n/2) + n$. Utilizando o método iterativo, temos.

\subsubsection{Metodo da Árvore de Recursão}
Relação de Recorrência da quicksort recursiva para o melhor caso, $T(n) = 2T\left(\dfrac{n}{2}\right) + n$. Utilizando o método árvore de recorrência.

\begin{table}[ht!]
    \centering
    \begin{tabular}{|c|c|c|c|}
    \hline
    \textbf{Nível} & \textbf{Tamanho} & $\#$\textbf{Nós} & \textbf{Tempo} \\ \hline
     0 & n & 1 & n \\ \hline
     1 & $n/2$ & 2 & $n/2$ \\ \hline
     2 & $n/4$ & 4 & $n/4$ \\ \hline
     3 & $n/8$ & 8 & $n/8$ \\ \hline
     (...) & & & \\ \hline
     i & $n/2^i$ & $2^i$ & $n/2^i$ \\ \hline 
    \end{tabular}  
    \caption{Árvore de recursão}
\end{table}

Calculando o somatório do tempo x nós.

\[
n/2^i = 1 \implies n = 2^i \implies i = \log_2{n}
\]

\[
\sum_{i=0}^{\log_2{n}} \left(2^i \cdot \dfrac{n}{2^i}\right) = \sum_{i=0}^{\log_2{n}} n = n \log_2{n}
\]

Logo, ele é $\Theta (n \log{n})$

\bigskip

Relação de Recorrência da quicksort recursiva para o pior caso, $T(n) = T(n - 1) + n$. 

Utilizando o método árvore de recorrência.

\begin{table}[ht!]
    \centering
    \begin{tabular}{|c|c|c|c|}
    \hline
    \textbf{Nível} & \textbf{Tamanho} & $\#$\textbf{Nós} & \textbf{Tempo} \\ \hline
     0 & n & 1 & n \\ \hline
     1 & $n - 1$ & 1 & $n - 1$ \\ \hline
     2 & $n - 2$ & 1 & $n - 2$ \\ \hline
     3 & $n - 3$ & 1 & $n - 3$ \\ \hline
     (...) & & & \\ \hline
     i & $n - i$ & $1$ & $n - i$ \\ \hline 
    \end{tabular}  
    \caption{Árvore de recursão}
\end{table}

Calculando o somatório do tempo x nós. 
\[
1 = n -i \implies i = n - 1
\]

\[
\sum_{i = 0}^{n - 1} \left(i\right) = \frac{n \left(n + 1\right)}{2}= \frac{n^2 + n}{2}
\]

Logo, ele é $\Theta (n^2)$, pois o $2^n$ é a complexidade dominante entre $n^2$ e $n$.
\subsubsection{Método do Teorema Mestre}
Utilizando o método Mestre para a análise de $T(n)$, temos:

Tome \( a = 2 \), \( b = 2 \), \( k = 1 \). Como \( a = b^k \Leftrightarrow 2 = 2^1 \), então temos que \( T(n) \) é \( \Theta(n^k \log{n}) \Rightarrow \Theta(n \log{n}) \).

