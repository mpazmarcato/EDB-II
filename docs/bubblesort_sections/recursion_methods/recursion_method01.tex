\subsubsection{Método da Substituição}

Seja a função recursiva do pior caso do Bubble Sort. Utilizando o método da substituição, vamos mostrar que \( T(n) \leq c \cdot n^2 \):

\textbf{Caso Base:} Para \( n = 1 \):
\[
T(1) = 1
\]
Queremos mostrar que \( T(1) \leq c \cdot 1^2 \), então:
\[
1 \leq c
\]
Logo, a desigualdade é satisfeita para \( c \geq 1 \).

\textbf{Hipótese de Indução:} Suponha que para algum \( 1 < n \leq k \), temos \( T(k) \leq c \cdot k^2 \).

\textbf{Passo Indutivo:} Queremos mostrar que \( T(k + 1) \leq c \cdot (k + 1)^2 \).

Dada a recorrência \( T(k + 1) = T(k) + (k + 1) \), aplicamos a hipótese de indução:
\[
T(k + 1) = T(k) + (k + 1) \leq c \cdot k^2 + (k + 1)
\]

Agora, precisamos demonstrar que essa expressão é menor ou igual a \( c \cdot (k + 1)^2 \):
\[
c \cdot k^2 + (k + 1) \leq c \cdot (k + 1)^2
\]
Expanda o lado direito:
\[
c \cdot (k + 1)^2 = c \cdot (k^2 + 2k + 1) = c \cdot k^2 + 2c \cdot k + c
\]
Assim, a desigualdade torna-se:
\[
c \cdot k^2 + (k + 1) \leq c \cdot k^2 + 2c \cdot k + c
\]

Cancelando \( c \cdot k^2 \) de ambos os lados, obtemos:
\[
k + 1 \leq 2c \cdot k + c
\]
Isso é verdade para valores apropriados de \( c \); por exemplo, \( c \geq 1 \) é suficiente.

\textbf{Conclusão:} Portanto, mostramos que \( T(n) = O(n^2) \) no pior caso para o Bubble Sort usando o método da substituição.
