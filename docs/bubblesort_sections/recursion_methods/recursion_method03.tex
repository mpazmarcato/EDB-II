\subsubsection{Método da Árvore de Recursão}

Relação de Recorrência do \textbf{Bubble Sort Recursivo} para o pior caso, \( T(n) = T(n - 1) + n \). Utilizando o método árvore de recorrência.

\begin{table}[ht!]
    \centering
    \begin{tabular}{|c|c|c|c|}
    \hline
    \textbf{Nível} & \textbf{Tamanho} & $\#$\textbf{Nós} & \textbf{Tempo} \\ \hline
     0 & n & 1 & n \\ \hline
     1 & $n - 1$ & 1 & $n - 1$ \\ \hline
     2 & $n - 2$ & 1 & $n - 2$ \\ \hline
     3 & $n - 3$ & 1 & $n - 3$ \\ \hline
     (...) & & & \\ \hline
     i & $n - i$ & 1 & $n - i$ \\ \hline 
    \end{tabular}  
    \caption{Árvore de recursão do Bubble Sort Recursivo}
\end{table}

\textbf{Calculando o somatório do tempo em cada nível}

O trabalho realizado em cada nível \(i\) é proporcional ao tamanho \( n - i \), com apenas um nó em cada nível. O somatório total do tempo em cada nível é:

\[
\sum_{i=0}^{n-1} (n - i) = n + (n-1) + (n-2) + \dots + 1
\]

Esta soma pode ser simplificada como a soma de uma progressão aritmética:

\[
T(n) = \frac{n(n+1)}{2}
\]

Logo, a complexidade final é:

\[
T(n) = O(n^2)
\]

Portanto, a complexidade do Bubble Sort recursivo no pior caso é \( O(n^2) \), dado que a quantidade de trabalho em cada nível é linear, e há \( n \) níveis de recursão.
