\subsubsection{Método da iteração}
Seja a função recursiva do pior caso do bubblesort definida por \( T(n) = T(n-1) + n \). Utilizando o método da iteração, vamos encontrar seu nível de complexidade. 

Expandindo a recorrência:
\begin{align*}
i = 1 : &T(n) = T(n - 1) + n \\
i = 2 : &T(n) = (T(n - 2) + (n - 1)) + n \\
i = 3 : &T(n) = (T(n - 3) + (n - 2) + (n - 1)) + n \\
i : & T(n) = T(n - i) + \dots + (n - 1) + n
\end{align*}

Assim, temos:

\[
T(n) = 1 + \sum_{k=2}^{n} k = 1 + \left( \frac{n(n-1)}{2} - 1 \right)
\]
\[
= \frac{n(n-1)}{2}
\]
\[
= \frac{n^2 - n}{2}
\]

Note que a complexidade dominante é \( n^2 \), portanto, a relação de recorrência tem complexidade de \( O(n^2) \).
