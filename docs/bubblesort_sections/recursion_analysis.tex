\subsection{Análise do Algoritmo Recursivo}
O algoritmo recursivo tem como função de complexidade $T(n) = T(n - 1) + n$ tanto para o melhor quanto para o pior caso.
A cada chamada da função, é esperado que ao menos um elemento seja ordenado corretamente, ocupando a última posição na lista atual.
Após isso, a função é chamada novamente, dessa vez passando somente o restante da lista que ainda não está ordenado, ou seja, a lista de $n - 1$ elementos. \\
Resolveremos a recorrência pelos métodos abaixo:
\subsubsection{Método da Substituição}

Pelo método da substituição, mostremos que $T(n)$ é $O(n \cdot \log n)$ para o melhor caso, ou seja, que $T(n) \le c \cdot n \log_2 n + d$: \\
\textbf{Demonstração.} \\
Seja $T(n)$ em seu melhor caso. \\
Caso base: $n = 2, T(2) = 2$:
\begin{align*} 
  T(n) \le c \cdot n \log_2 n + d & \implies T(2) \le c \cdot (2 \log_2 2) + d \\ 
  &\implies 2 \le c \cdot 2 + d\\ 
  &\implies c + \frac{d}{2} \ge 1 
\end{align*}
Portanto, o caso base é verdadeiro. \\
Hipótese Indutiva: Suponha um $k \in \mathbb{N}$ arbitrário, tal que $T(k) \le c \cdot (k \cdot \log_2 k) + d$. \\
Agora, note que $\frac{k}{2} < k, \forall k > 2$. Então, temos:
\begin{align*}
  T(k) \le 2T(\frac{k}{2}) + k  & \implies 2T(\frac{k}{2}) + k \le 2 \cdot (c \cdot \frac{k}{2} \cdot \log_2 \frac{k}{2} + d) + k & \quad \text{(Por H.I)} \\
  & \implies 2 \cdot (c \cdot \frac{k}{2} \cdot \log_2 \frac{k}{2} + d) + k  \le c \cdot k \cdot \log_2 k + d \\
  & \implies ck \cdot \log_2 \frac{k}{2} + 2d + k \le ck \cdot \log_2 k + d \\
  & \implies ck \cdot (\log_2 k - \log_2 2) + 2d + k \le ck \cdot \log_2 k + d & \quad \text{(Por prop. de $\log$)} \\
  & \implies ck\log_2 k - ck  + 2d + k \le ck \cdot \log_2 k + d & \quad \text{($log_a a = 1$)} \\
  & \implies k - ck  + 2d \le d \\
  & \implies k - ck \le -d \\
  & \implies ck - k \ge d \\
\end{align*}
Assuma que $d = ck - k$. Logo, temos:
\begin{align*}
  c + \frac{d}{2} \ge 1 & \implies c + \frac{ck - k}{2} \ge 1 \\
  & \implies c + \frac{2c - 2}{2} \ge 1 & \quad \text{(Assuma $k = 2$.)} \\
  & \implies c + c - 1 \ge 1 \\
  & \implies 2c \ge 2 \\
  & \implies c \ge 1 \\
\end{align*}
Portanto, vale que o melhor caso para o Quick Sort recursivo é $O(n \log_2 n)$. \\
Então, mostremos que $T(n)$ é $O(n^2)$ para o pior caso, ou seja, que $T(n) \le c \cdot n^2$. \\
\textbf{Demonstração.} \\
Seja $T(n)$ em seu pior caso. \\
Caso base: $n = 1, T(1) = 1$:
\begin{align*}
  T(n) \le c \cdot n^2 & \implies T(1) \le c \cdot 1^2 \\
  & \implies c \ge 1 
\end{align*}

Portanto, o caso base é verdadeiro. \\
Hipótese Indutiva: Suponha um $k \in \mathbb{N}$ arbitrário, tal que $T(k) \le c \cdot k^2$. \\
Agora, note que $k - 1 \le k, \forall k > 1$. Logo, temos:
\begin{align*}
  T(k) \le T(k - 1) + k & \implies T(k - 1) + k \le c \cdot (k-1)^2 + k & \quad \text{(Por H.I)} \\
  & \implies c \cdot (k-1)^2 + k \le c \cdot k^2  \\
  & \implies c \cdot (k^2 - 2k + 1) + k \le c \cdot k^2  \\
  & \implies ck^2 - 2ck + c + k \le c \cdot k^2  \\
  & \implies c - 2ck + k \le 0  \\
  & \implies k - 2ck \le -c  \\
  & \implies k(1 - 2c) \le -c  \\
  & \implies 2(1 - 2c) \le -c &\quad \text{(Assuma $k = 2$)} \\
  & \implies 2 - 4c\le -c \\
  & \implies 1 - \frac{4c}{2} \le \frac{-c}{2} &\quad \text{(Divisão por 2)} \\
  & \implies 1 \le \frac{-c}{2} + \frac{4c}{2}\\
  & \implies 1 \le \frac{3c}{2} \\
  & \implies 1 \approx \frac{2}{3} \le c \\
\end{align*}

Portanto, vale que o pior caso do Quick Sort recursivo é $O(n^2)$.

\subsubsection{Metodo da Árvore de Recursão}
Relação de Recorrência da quicksort recursiva para o melhor caso, $T(n) = 2T(n/2) + n$. Utilizando o método árvore de recorrência.

\begin{table}[ht!]
    \centering
    \begin{tabular}{|c|c|c|c|}
    \hline
    \textbf{Nível} & \textbf{Tamanho} & $\#$\textbf{Nós} & \textbf{Tempo} \\ \hline
     0 & n & 1 & n \\ \hline
     1 & $n/2$ & 2 & $n/2$ \\ \hline
     2 & $n/4$ & 4 & $n/4$ \\ \hline
     3 & $n/8$ & 8 & $n/8$ \\ \hline
     (...) & & & \\ \hline
     i & $n/2^i$ & $2^i$ & $n/2^i$ \\ \hline 
    \end{tabular}  
    \caption{Árvore de recursão}
\end{table}

Calculando o somatório do tempo x nós. \\
$n/2^i = 1 \rightarrow n = 2^i \rightarrow i = \log_2{n}$ \\
$\sum_{i=0}^{\log_2{n}} (2^i \times n/2^i) = \sum_{i=0}^{\log_2{n}} n = n \log_2{n}$. Logo, ele é $\Theta n \log{n}$ 

Relação de Recorrência da quicksort recursiva para o pior caso, $T(n) = T(n - 1) + n$. Utilizando o método árvore de recorrência.

\begin{table}[ht!]
    \centering
    \begin{tabular}{|c|c|c|c|}
    \hline
    \textbf{Nível} & \textbf{Tamanho} & $\#$\textbf{Nós} & \textbf{Tempo} \\ \hline
     0 & n & 1 & n \\ \hline
     1 & $n - 1$ & 1 & $n/2$ \\ \hline
     2 & $n - 2$ & 1 & $n/4$ \\ \hline
     3 & $n - 3$ & 1 & $n/8$ \\ \hline
     (...) & & & \\ \hline
     i & $n - i$ & $1$ & $n/2^i$ \\ \hline 
    \end{tabular}  
    \caption{Árvore de recursão}
\end{table}

Calculando o somatório do tempo x nós. \\
$1 = n -i \rightarrow i = n - 1$ \\
$\sum_{i = 0}^{n - 1} (n/2^i) = 2n(1 - 1/2^n)$. Logo, ele é $\Theta n$, pois o $2^n$ está tendendo a 0, tornando o n a complexidade predominante.

\subsubsection{Método do Teorema Mestre}
O método do teorema mestre não é aplicável ao algoritmo do Bubble Sort, já que sua função de complexidade $T(n)$ não está no padrão $aT(\frac{n}{b}) + \Theta(n^k)$.

