\subsection{Análise do Algoritmo Iterativo}
O algoritmo iterativo tem diferentes funções de complexidade para cada caso:
\begin{itemize}
  \item Melhor caso: ocorre quando a lista está ordenada; o loop interno será executado $n$ vezes e o externo, somente 1. Portanto, temos uma complexidade expressa por algo como $f(n) = n + 3$, onde $n$ representa o loop \textit{for} e o 3, as instruções que não dependem da entrada.
  \item Pior caso: ocorre quando a lista está completamente desordenada; o loop externo ocorre $n$ vezes e também o interno. Desse modo, temos uma complexidade expressa por algo como $g(n) = n^2 + 5$, onde 5 representa o total de instruções que não dependem da entrada e $n^2$ a quantidade de iterações total feito pelos loops.
\end{itemize}

Note que, para o melhor caso, o algoritmo tem complexidade $\Omega(n)$, pois:
\begin{align*}
    \lim_{n\to\infty} \frac{f(n)}{n} &= \lim_{n\to\infty} \frac{n + 3}{n} \\
    &= \lim_{n\to\infty} 1 + \frac{3}{n} \\ 
    &= 1 
\end{align*}

Adicionalmente, para o pior caso, o algoritmo tem complexidade $O(n^2)$, pois:
\begin{align*}
    \lim_{n\to\infty} \frac{g(n)}{n} &= \lim_{n\to\infty} \frac{n^2 + 5}{n^2} \\
    &= \lim_{n\to\infty} 1 + \frac{5}{n^2} \\ 
    &= 1 
\end{align*}

Portanto, o algoritmo tem $\Omega(n)$, $O(n^2)$ e não há $\Theta$.
