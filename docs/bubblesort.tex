\section{BubbleSort}
Aqui estão os pseudocódigos do algoritmo de \textbf{BubbleSort}, em sua versão iterativa e recursiva.
\subsection{Iterativa}
\begin{verbatim}
# Realiza a ordenação de um vetor desordenado com a estratégia do MergeSort.
mergeSort(vetor) {
    n = tamanho(vetor) # Obtém o tamanho do vetor
    largura = 1 # Inicializa a largura das sub-listas

    # Enquanto a largura das sub-listas for menor que o tamanho do vetor
    enquanto (largura < n) {
        # Itera sobre o vetor em passos de 2 * largura
        para (int i = 0; i < n - 1; i += 2 * largura) {
            inicio = i # Início da sub-lista
            meio = min(i + largura - 1, n - 1) # Meio da sub-lista
            fim = min(i + 2 * largura - 1, n - 1) # Fim da sub-lista

            # Chama a função de mesclagem para combinar as sub-listas
            merge(vetor, inicio, meio, fim)
        }

        # Dobra a largura para a próxima iteração
        largura = largura * 2
    }
}

# Função para mesclar duas sub-listas
funcao merge(vetor, inicio, meio, fim) {
    n1 = meio - inicio + 1 # Tamanho da primeira sub-lista
    n2 = fim - meio # Tamanho da segunda sub-lista

    # Vetores auxiliares para armazenar as sub-listas
    L: vetor
    R: vetor

    # Copia os elementos para o vetor auxiliar L
    para (int i = 0; i < n1; i++) {
        L[i] = vetor[inicio + i]
    }

    # Copia os elementos para o vetor auxiliar R
    para (int i = 0; i < n2; i++) {
        R[j] = vetor[meio + 1 + j]
    }

    i = 0 # Índice para L
    j = 0 # Índice para R
    k = inicio # Índice para vetor

    # Mescla as duas sub-listas em vetor
    enquanto (i < n1 && j < n2) {
        se (L[i] <= R[j]) {
            vetor[k] = L[i] # Adiciona o menor elemento em vetor
            i = i + 1 # Move para o próximo elemento em L
        } senão {
            vetor[k] = R[j] # Adiciona o menor elemento em vetor
            j = j + 1 # Move para o próximo elemento em R
        }
        k = k + 1 # Move para o próximo índice em vetor
    }

    # Copia os elementos restantes de L, se houver
    enquanto (i < n1) {
        vetor[k] = L[i]
        i = i + 1
        k = k + 1
    }

    # Copia os elementos restantes de R, se houver
    enquanto (j < n2) {
        vetor[k] = R[j]
        j = j + 1
        k = k + 1
    }
}

\end{verbatim} 

\subsection{Análise dos Algoritmos}
O algoritmo iterativo tem diferentes funções de complexidade para cada caso:
\begin{itemize}
  \item Melhor caso: no melhor caso (o vetor já está ordenado), o loop \textit{for} será executado $n$ vezes e o \textit{do-while}, somente 1. Portanto, temos uma complexidade expressa por algo como $f(n) = n + 3$, onde $n$ representa o loop \textit{for} e o 3, as instruções que não dependem da entrada: o $do-while$, a atribuição da variável \textit{conflito} e a verificação dentro do loop.
  \item Pior caso: no pior caso, o loop externo ocorre $n$ vezes e também o interno. Desse modo, temos uma complexidade expressa por algo como $g(n) = n^2 + 5$, onde 5 representa o total de instruções que não dependem da entrada e $n^2$ a quantidade de iterações total feito pelos loops.
\end{itemize}

Logo, temos que o algoritmo iterativo é $O(n^2)$ e $\Omega(n)$, não existindo $\Theta$.
