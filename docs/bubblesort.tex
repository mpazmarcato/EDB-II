\section{BubbleSort}
Aqui estão os pseudocódigos do algoritmo de \textbf{BubbleSort}, em sua versão iterativa e recursiva.
\subsection{Iterativa}
\begin{verbatim}

# Realiza a ordenação de um vetor desordenado com o algoritmo Quick Sort de forma iterativa.
quickSortIterative(vetor, valorEsquerda, valorDireita) {
    # Inicializa uma pilha para gerenciar os subvetores
    pilha = nova Pilha()
    
    # Empilha os limites iniciais do vetor
    pilha.empilhar((valorEsquerda, valorDireita))
    
    # Processa os subvetores até que a pilha esteja vazia
    enquanto (pilha não está vazia) {
        # Desempilha os limites atuais do subvetor
        (valorBaixo, valorAlto) = pilha.desempilhar()
        
        # Particiona o subvetor atual e obtém a posição do pivô
        pivo = particiona(vetor, valorBaixo, valorAlto)
        
        # Se houver elementos à esquerda do pivô, empilha o subvetor esquerdo
        se (valorBaixo < pivo - 1) {
            pilha.empilhar((valorBaixo, pivo - 1))
        }
        
        # Se houver elementos à direita do pivô, empilha o subvetor direito
        se (pivo < valorAlto) {
            pilha.empilhar((pivo, valorAlto))
        }
    }
}

# Função para particionar um vetor em dois baseado em um pivô.
particiona(vetor, valorBaixo, valorAlto) { 
  # Obter o valor central relativo ao vetor.
  meio = valorBaixo + (valorAlto - valorBaixo) / 2
  pivo = vetor[meio] # Obtém o pivô.

  # Variáveis internas que controlam os índices em uso.
  inicio = valorBaixo
  final = valorAlto 
  
  # Percorre de uma extremidade à outra.
  enquanto(inicio <= final) { 
    # Ajusta o índice da extremidade esquerda até encontrar um elemento 
    # à esquerda maior que o pivô.
    enquanto (vetor[inicio] < pivo) { 
      inicio++
    }
    # Ajusta o índice da extremidade direita até encontrar um elemento 
    # à direita menor que o pivô.
    enquanto (vetor[final] > pivo) {
      final--
    }

    # Se os índices das extremidades não se encontraram, 
    # os valores maiores e menores que o pivô não estão
    # separados, fazendo necessária uma troca arbitrária.
    se (inicio <= final) {
      trocar(vetor, inicio, final)
      inicio++
      final--
    }
  }
  
  # Retorna a posição do novo pivô
  retorne inicio
}
   
# Função para trocar dois elementos em um vetor.
trocar(vetor, indice1, indice2) { 
  elementoAuxiliar = vetor[indice2]
  vetor[indice2] = vetor[indice1]
  vetor[indice1] = elementoAuxiliar
}
    \end{verbatim} 

\subsection{Análise dos Algoritmos}
O algoritmo iterativo tem diferentes funções de complexidade para cada caso:
\begin{itemize}
  \item Melhor caso: no melhor caso (o vetor já está ordenado), o loop \textit{for} será executado $n$ vezes e o \textit{do-while}, somente 1. Portanto, temos uma complexidade expressa por algo como $f(n) = n + 3$, onde $n$ representa o loop \textit{for} e o 3, as instruções que não dependem da entrada: o $do-while$, a atribuição da variável \textit{conflito} e a verificação dentro do loop.
  \item Pior caso: no pior caso, o loop externo ocorre $n$ vezes e também o interno. Desse modo, temos uma complexidade expressa por algo como $g(n) = n^2 + 5$, onde 5 representa o total de instruções que não dependem da entrada e $n^2$ a quantidade de iterações total feito pelos loops.
\end{itemize}

Logo, temos que o algoritmo iterativo é $O(n^2)$ e $\Omega(n)$, não existindo $\Theta$.
