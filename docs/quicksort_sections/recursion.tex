\subsection{Recursiva}
\begin{verbatim}

# Realiza a ordenação de um vetor com o algoritmo Quick Sort, 
# recebendo como entrada o vetor e os índices das extremidades.
quicksort(vetor, valorEsquerda, valorDireita) { 
  if (valorEsquerda < valorDireita) {
    # Pivo recebe o ponto onde o vetor será particionado.
    pivo = particiona(vetor, valorEsquerda, valorDireita)
    # Realiza a ordenação no subvetor à esquerda do ponto determinado.
    quicksort(vetor, valorEsquerda, pivo)
    # Realiza a ordenação no subvetor à direita do ponto determinado.
    quicksort(vetor, pivo + 1, valorDireita)
  }
}

# Função para particionar um vetor em dois baseado em um pivô.
particiona(vetor, valorBaixo, valorAlto) { 
  # Obter o valor central relativo ao vetor.
  meio = valorBaixo + (valorAlto - valorBaixo) / 2 

  # O pivô é o elemento central do vetor analisado.
  pivo = vetor[meio] 

  # Variáveis internas que controlam os índices em uso.
  inicio = valorBaixo
  final = valorAlto 
  
  # Percorre de uma extremidade à outra.
  enquanto(inicio <= final) { 
    # Ajusta o índice da extremidade esquerda até encontrar um elemento 
    # à esquerda maior que o pivô.
    enquanto (vetor[inicio] < pivo) { 
      inicio++
    }
    # Ajusta o índice da extremidade direita até encontrar um elemento 
    # à direita menor que o pivô.
    enquanto (vetor[final] > pivo) {
      final--
    }

    # Se os índices das extremidades não se encontraram, 
    # os valores maiores e menores que o pivô não estão
    # separados, fazendo necessária uma troca arbitrária.
    se (inicio <= final) {
      trocar(vetor, inicio, final)
      inicio++
      final--
    }
  }
  
  # Retorna a posição do novo pivô
  retorne inicio
}
   
# Função para trocar dois elementos em um vetor.
trocar(vetor, indice1, indice2) { 
  elementoAuxiliar = vetor[indice2]
  vetor[indice2] = vetor[indice1]
  vetor[indice1] = elementoAuxiliar
}
\end{verbatim}

