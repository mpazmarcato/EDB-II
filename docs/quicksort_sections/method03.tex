\subsection{Metodo 03 - Método Árvore de Recursão}
Relação de Recorrência da quicksort recursiva para o melhor caso, $T(n) = 2T(n/2) + n$. Utilizando o método árvore de recorrência.

\begin{table}[ht!]
    \centering
    \begin{tabular}{|c|c|c|c|}
    \hline
    \textbf{Nível} & \textbf{Tamanho} & $\#$\textbf{Nós} & \textbf{Tempo} \\ \hline
     0 & n & 1 & n \\ \hline
     1 & $n/2$ & 2 & $n/2$ \\ \hline
     2 & $n/4$ & 4 & $n/4$ \\ \hline
     3 & $n/8$ & 8 & $n/8$ \\ \hline
     (...) & & & \\ \hline
     i & $n/2^i$ & $2^i$ & $n/2^i$ \\ \hline 
    \end{tabular}  
    \caption{Árvore de recursão}
\end{table}

Calculando o somatório do tempo x nós. \\
$n/2^i = 1 \rightarrow n = 2^i \rightarrow i = \log_2{n}$ \\
$\sum_{i=0}^{\log_2{n}} (2^i \times n/2^i) = \sum_{i=0}^{\log_2{n}} n = n \log_2{n}$. Logo, ele é $\Theta n \log{n}$ 

Relação de Recorrência da quicksort recursiva para o pior caso, $T(n) = T(n - 1) + n$. Utilizando o método árvore de recorrência.

\begin{table}[ht!]
    \centering
    \begin{tabular}{|c|c|c|c|}
    \hline
    \textbf{Nível} & \textbf{Tamanho} & $\#$\textbf{Nós} & \textbf{Tempo} \\ \hline
     0 & n & 1 & n \\ \hline
     1 & $n - 1$ & 1 & $n/2$ \\ \hline
     2 & $n - 2$ & 1 & $n/4$ \\ \hline
     3 & $n - 3$ & 1 & $n/8$ \\ \hline
     (...) & & & \\ \hline
     i & $n - i$ & $1$ & $n/2^i$ \\ \hline 
    \end{tabular}  
    \caption{Árvore de recursão}
\end{table}

Calculando o somatório do tempo x nós. \\
$1 = n -i \rightarrow i = n - 1$ \\
$\sum_{i = 0}^{n - 1} (n/2^i) = 2n(1 - 1/2^n)$. Logo, ele é $\Theta n$, pois o $2^n$ está tendendo a 0, tornando o n a complexidade predominante.