<<<<<<<< HEAD:docs/quicksort_sections/recursion_methods/recursion_method04.tex
\subsubsection{Método do Teorema Mestre}
========
\subsubsection{Método 04 - Método Mestre}
>>>>>>>> 9e4497fefbd53b11f69e25bf7f167bdcd9e3f9d4:docs/quicksort_sections/recursion_method04.tex
Relação de Recorrência da quicksort recursiva em seu melhor caso, $T(n) = 2T(n/2) + n$. Utilizando o método Mestre, temos.
Tome a = 2, b = 2, k = 1. Como $a = b^k \Leftrightarrow 2 = 2^1$, então o T(n) é $\Theta(n^k \log{n}) \Rightarrow \Theta(n \log{n})$. \\
O método mestre não é aplicavel para o seu pior caso, pois ele é, $T(n) = T(n-1) + n$.
