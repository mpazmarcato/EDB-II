\subsubsection{Método do Teorema Mestre}

A relação de recorrência da quicksort recursiva em seu melhor caso é dada por \( T(n) = 2T(n/2) + n \). Utilizando o método mestre, temos:

Tome \( a = 2 \), \( b = 2 \), \( k = 1 \). Como \( a = b^k \Leftrightarrow 2 = 2^1 \), então \( T(n) \) é \( \Theta(n^k \log{n}) \Rightarrow \Theta(n \log{n}) \).

O método mestre não é aplicável para o seu pior caso, pois a relação é \( T(n) = T(n-1) + n \).
