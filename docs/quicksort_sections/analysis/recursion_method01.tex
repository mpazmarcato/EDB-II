\subsubsection{Método da Substituição}
Note que o algoritmo recursivo tem duas possíveis relações de recorrência para o melhor e pior caso. \\
\begin{itemize}
  \item Melhor caso: ocorre quando a escolha do pivô particiona o vetor em duas partes quase iguais, gerando a relação de recorrência dada por $T_1(n) = 2T(\frac{n}{2}) + \Theta(n)$.
  \item Pior caso: ocorre quando a escolha do pivô particiona o vetor de forma em um subvetor de tamanho mínimo e outro com todos os elementos, gerando a relação de recorrência dada por $T(n) = T(n - 1) + T(1) + \Theta(n)$.
\end{itemize}

Agora, mostremos que $T_1(n)$ é $O(n \cdot log n)$.

Então, mostremos que $T_2(n)$ é $O(n^2)$.
