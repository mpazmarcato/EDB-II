\subsection{Compute os tempos de processamento e compare}

\begin{table}[ht]
    \centering
    \caption{Resultados do Desempenho das Funções `idadeRep` e `idadeRep2`.}
    \begin{tabular}{@{}ccc@{}}
        \toprule
        Tamanho do Vetor (\(n\)) & Tempo `idadeRep` (µs) & Tempo `idadeRep2` (µs) \\ \midrule
        100       & 1.842               & 33.051               \\
        1000      & 11.589              & 400.563              \\
        10000     & 126.964             & 2260.4               \\ \bottomrule
    \end{tabular}
\end{table}

\begin{itemize}
    \item Análise de Desempenho das Funções \texttt{idadeRep} e \texttt{idadeRep2}
\end{itemize}
Através do teste de desempenho nas duas funções, \texttt{idadeRep} e \texttt{idadeRep2}, fomos verificar a presença de idades duplicadas em um vetor de inteiros variando de 0 a 100. Os testes foram conduzidos em três tamanhos diferentes de vetores: \( n = 100 \), \( n = 1000 \), e \( n = 10000 \). Os resultados foram medidos em microsegundos e revelaram diferenças significativas no desempenho entre as duas abordagens.

\begin{itemize}
    \item Resultados Obtidos
\end{itemize}
Os resultados foram os seguintes:

\begin{enumerate}
    \item Para \( n = 100 \):
    \begin{itemize}
        \item \texttt{idadeRep}: 1.842 microsegundos
        \item \texttt{idadeRep2}: 33.051 microsegundos
    \end{itemize}

    \item Para \( n = 1000 \):
    \begin{itemize}
        \item \texttt{idadeRep}: 11.589 microsegundos
        \item \texttt{idadeRep2}: 400.563 microsegundos
    \end{itemize}

    \item Para \( n = 10000 \):
    \begin{itemize}
        \item \texttt{idadeRep}: 126.964 microsegundos (aproximadamente 0.127 milissegundos)
        \item \texttt{idadeRep2}: 2260.4 microsegundos (aproximadamente 2.260 milissegundos)
    \end{itemize}
\end{enumerate}

\begin{itemize}
    \item Análise das Funções
\end{itemize}
A função \texttt{idadeRep} possui uma complexidade de O(n), o que indica que seu tempo de execução aumenta linearmente com o tamanho do vetor. Isso foi evidente nos resultados, onde o tempo de execução cresceu de forma relativamente linear à medida que o tamanho do vetor aumentou.

Por outro lado, a função \texttt{idadeRep2} apresenta uma complexidade de O(n log n), devido à operação de ordenação (\texttt{std::sort}). Isso resultou em um tempo de execução que cresceu mais rapidamente em comparação com \texttt{idadeRep}, especialmente à medida que \( n \) aumentou. Os resultados mostram que, para vetores pequenos (\( n = 100 \)), a diferença de desempenho não é tão pronunciada, mas para vetores maiores (\( n = 10000 \)), a função \texttt{idadeRep} foi significativamente mais rápida.

\begin{itemize}
    \item Conclusões
\end{itemize}
A análise revela que a função \texttt{idadeRep} é mais eficiente em termos de tempo de execução, sendo mais adequada para cenários em que a rapidez na verificação de duplicatas é crucial. A função \texttt{idadeRep2}, apesar de funcional, sofre um aumento significativo no tempo de execução devido ao custo da ordenação, tornando-se impraticável para grandes conjuntos de dados.

Em resumo, ao lidar com grandes vetores, é essencial escolher algoritmos com complexidade mais baixa, como o da função \texttt{idadeRep}, para garantir um desempenho adequado e eficiente nas operações desejadas. Essa escolha é fundamental em aplicações onde a eficiência computacional é uma prioridade.
