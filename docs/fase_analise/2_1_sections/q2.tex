\subsection{Crie instâncias aleatórias com idades (inteiros) variando entre 0 e 100, com tamanhos \( n = 100 \), \( n = 1000 \), \( n = 1000000 \)}

\begin{verbatim}
#include <iostream>
#include <vector>
#include <algorithm>
#include <random>
#include <chrono>

std::vector<int> generateRandomAges(int size) {
    std::vector<int> ages(size);
    std::random_device rd;
    std::mt19937 gen(rd());
    std::uniform_int_distribution<> dis(1, 100);

    for (int i = 0; i < size; ++i) {
        ages[i] = dis(gen);
    }

    return ages;
}

int main() {
    std::vector<int> sizes = {100, 1000, 10000};

    for (int size : sizes) {
        std::vector<int> ages = generateRandomAges(size);

        // Medindo o tempo para idadeRep
        auto start1 = std::chrono::high_resolution_clock::now();
        bool result1 = idadeRep(ages);
        auto end1 = std::chrono::high_resolution_clock::now();
        std::chrono::duration<double> duration1 = end1 - start1;

        // Medindo o tempo para idadeRep2
        auto start2 = std::chrono::high_resolution_clock::now();
        bool result2 = idadeRep2(ages);
        auto end2 = std::chrono::high_resolution_clock::now();
        std::chrono::duration<double> duration2 = end2 - start2;

        std::cout << "For vector size " << size << ":\n";
        std::cout << "idadeRep: " << (result1 ? "repeated" : "no repeat") 
                  << ", time taken: " << duration1.count() << " seconds\n";
        std::cout << "idadeRep2: " << (result2 ? "repeated" : "no repeat") 
                  << ", time taken: " << duration2.count() << " seconds\n";
        std::cout << "-------------------------------------\n";
    }

    return 0;
}


\end{verbatim}
