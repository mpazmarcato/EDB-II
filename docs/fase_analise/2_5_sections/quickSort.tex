\subsubsection{Quicksort}

\hspace{0.6cm}O Quicksort se destaca como um algoritmo extremamente eficaz para conjuntos de até 100k elementos, com sua versão recursiva apresentando um tempo de execução de aproximadamente 1 ms em listas de 10k elementos. Essa eficiência pode ser atribuída à sua complexidade média de tempo de \(O(n \log n)\), que resulta da divisão recursiva da lista em sublistas menores, permitindo que o algoritmo processe os dados de forma mais rápida em comparação a métodos menos eficientes.

A diferença de desempenho entre as versões recursiva e iterativa se torna evidente em conjuntos maiores. A versão recursiva é mais rápida a partir de 10k elementos, enquanto para conjuntos menores, ambas as versões têm tempos de execução próximos de 0 ms. Isso se deve ao fato de que a sobrecarga das chamadas recursivas é mínima em listas pequenas, permitindo que o Quicksort opere rapidamente, mesmo em suas implementações iterativa e recursiva.

Entretanto, é importante considerar que, no pior caso, a complexidade do Quicksort pode chegar a \(O(n^2)\), o que ocorre em situações em que a lista já está ordenada ou quase ordenada. Essa situação se deve à escolha de pivô ineficiente, levando a uma divisão desigual das sublistas. Para mitigar esse problema, é comum implementar estratégias de escolha de pivô, como o pivô aleatório ou o método do "pivô mediano", que ajudam a manter o desempenho do algoritmo em \(O(n \log n)\).

Assim, o Quicksort é versátil e eficiente, sendo uma escolha recomendada para a ordenação de grandes conjuntos de dados, especialmente a sua versao recursiva devido ao seu desempenho superior em comparação à iterativa.
