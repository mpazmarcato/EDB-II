\subsection{Conclusão}

\hspace{0.6cm}Em geral, a escolha do algoritmo de ordenação ideal depende do tamanho do conjunto de dados e da exigência de eficiência. Para listas pequenas, a diferença de desempenho entre os algoritmos Bubblesort, Mergesort e Quicksort é menos significativa. No entanto, à medida que o volume de dados aumenta, o impacto da complexidade de cada algoritmo se torna evidente. 

O Bubblesort rapidamente se torna impraticável devido ao seu tempo de execução excessivo, que cresce quadraticamente com o aumento do número de elementos. Em contraste, tanto o Mergesort quanto o Quicksort mantêm tempos de execução significativamente inferiores, o que os torna mais adequados para conjuntos de dados maiores. Em particular, o Quicksort se destaca em cenários que envolvem grandes volumes de dados, aproveitando sua velocidade na versão recursiva, que frequentemente o torna a escolha mais eficaz para ordenação em menos tempo. 

Assim, a análise comparativa dos algoritmos evidencia que a seleção do método de ordenação deve levar em consideração não apenas o tamanho dos dados, mas também o desempenho esperado, garantindo uma abordagem otimizada para a ordenação em diferentes contextos.
