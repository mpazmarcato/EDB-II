\subsubsection{Bubblesort}

\hspace{0.6cm}O Bubblesort, tanto em sua versão iterativa quanto recursiva, não é recomendado para listas com mais de 10 mil elementos. Durante os testes, a versão iterativa levou 49676 ms e a recursiva, 36172 ms, apresentando uma diferença de desempenho mínima. Ambas as implementações são lentas devido à complexidade de tempo \(O(n^2)\), tornando o Bubblesort ineficaz para grandes conjuntos de dados.

O Bubblesort funciona realizando múltiplas passagens pela lista e comparando pares de elementos adjacentes, trocando-os se estiverem na ordem errada. Esse processo continua até que a lista esteja ordenada. Embora sua implementação seja simples e intuitiva, a ineficiência do algoritmo se torna evidente em conjuntos grandes, onde o número de comparações e trocas aumenta quadraticamente.

Portanto, o Bubblesort é mais adequado para listas pequenas, onde sua simplicidade e facilidade de implementação podem ser mais vantajosas.
